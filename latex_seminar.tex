\documentclass{beamer}

%\usepackage{handoutWithNotes}
%\pgfpagesuselayout{3 on 1 with notes}[letterpaper,border shrink=5mm]
\usepackage[semibold]{raleway} % Use raleway font for \sffamily
\usepackage[protrusion=true,expansion=true,kerning=true,spacing=true,tracking=true,final]{microtype}
\usepackage{graphicx} % Allows including images
\usepackage{booktabs} % Allows the use of \toprule, \midrule and \bottomrule in tables
\usepackage[outputdir=build]{minted} % for code highlighting
\usemintedstyle{igor}


\mode<presentation> {%
\usetheme{simple} % choose simple theme
}


%-----------------------------------------------------------------------------
%	TITLE PAGE
%-----------------------------------------------------------------------------
\setwatermark[hoffset=6.75cm,voffset=7.65cm]{\includegraphics[height=1.50cm]{Rockin_R_Initials-4C.png}}

\title[Thesis with \LaTeX]{Writing a Masters Thesis with \LaTeX} % The short title appears at the bottom of every slide, the full title is only on the title page

\author{%
  Guillermo Garza\\[2pt]% Your name
 \footnotesize\textit{guillermo.garza@utrgv.edu} % Your email address
}

\institute[UTRGV] % Your institution as it will appear on the bottom of every slide, may be shorthand to save space
{The University of Texas Rio Grande Valley}

\date{\today} % Date, can be changed to a custom date


\begin{document}



\begin{frame}
  \titlepage
\end{frame}

\setwatermark[hoffset=0cm,voffset=8.25cm]{\includegraphics[height=0.9cm]{Rockin_R_Initials-4C.png}}

\begin{frame}
\frametitle{Overview} % Table of contents slide, comment this block out to remove it
\tableofcontents % Throughout your presentation, if you choose to use \section{} and \subsection{} commands, these will automatically be printed on this slide as an overview of your presentation
\end{frame}

%----------------------------------------------------------------------------
%	PRESENTATION SLIDES
%------------------------------------------------

\section{What is \LaTeX{} and Why Should I Use It?}
%------------------------------------------------

\begin{frame}
\frametitle{What is \LaTeX{} and\\ Why Should I Use It?}

\begin{itemize}
  \item
    \LaTeX{} is a \textbf{typesetting language}.  It is used to prepare all sorts of
    documents.
  \item
    It is the standard language use by nearly all mathematicians.
  \item
     When used properly, it separates content and style.
\end{itemize}


\end{frame}



\section{Getting Started with \LaTeX}
%------------------------------------------------

\begin{frame}
\frametitle{Getting Started with \LaTeX}
%------------------------------------------------

  The easiest way to get started with \LaTeX{} is by signing up with a service
  such as Overleaf or Sharelatex.

  \begin{itemize}
    \item
      \href{http://www.overleaf.com}{www.overleaf.com}
    \item
      \href{http://www.sharelatex.com}{www.sharelatex.com}
  \end{itemize}

  For faster compilation, download a \LaTeX{} distribution such as TeX Live.
  Along with a \LaTeX{} editor such as TeXworks.
  \begin{itemize}
    \item
      \href{http://www.tug.org/texlive/}{www.tug.org/texlive/}
    \item
      \href{http://www.tug.org/texworks/}{www.tug.org/texworks/}
  \end{itemize}

\end{frame}


%------------------------------------------------
\section{UTRGV Thesis Template}
\begin{frame}
  \frametitle{UTRGV Thesis Template}
  The UTRGV Thesis Template is available at
  \url{https://github.com/UTRGV-SMSS/Thesis-Template}\\

  Download the zip file from there.  If you are using a service such
  as Overleaf,  upload the zip file to create a new project.

  \begin{itemize}
      \item
        Backup your files regularly.
      \item
        Redownload the template and update
        the UTRGVthesis.cls file before you submit your thesis for review.
      \item
        Report any issues you have with the template on that website.
      \item
        Get help from us.
  \end{itemize}

\end{frame}


%------------------------------------------------
\section{Useful References}
\begin{frame}
  \frametitle{Useful References}


  \begin{itemize}
    \item
      Introduction to \LaTeX{} on YouTube from GVSU Math\\
      \url{https://youtu.be/cTEfw-jUqAg}

    \item
      \LaTeX{} Wikibook\\
      \url{https://en.wikibooks.org/wiki/LaTeX}

    \item
      Cheatsheet
      \url{http://joshua.smcvt.edu/undergradmath/undergradmath.pdf}

    \item
      The Short Math Guide for \LaTeX{}\\
      \url{http://tex.loria.fr/general/downes-short-math-guide.pdf}\\
      is a good reference for typesetting math.


      \item
        Google and StackOverflow (prepend the word ``Latex'' to your search)
  \end{itemize}

\end{frame}


%------------------------------------------------
\section{Creating Tables}
\begin{frame}
  \frametitle{Creating Tables}

  Use the following website to generate \LaTeX{} tables:
  \url{http://www.tablesgenerator.com}\\

  My workflow is to keep an Excel file of table data. I copy that into the
  website, by clicking ``File'' then ``Paste Table Data''.\\

  It is a good idea to use the ``booktabs'' style.


\end{frame}


%------------------------------------------------
\section{Thesis Template with LyX}
\begin{frame}[fragile]{Thesis Template with LyX}

  We've created a rudimentary template LyX thesis template,
  that does not have many features, but can be used together with the
  thesis template.

  \url{https://github.com/UTRGV-SMSS/LyX_Thesis_Template}\\

  \begin{enumerate}
      \item
        Download and use LyX Thesis Template to write your thesis.
      \item
        Download the \LaTeX{} UTRGV Thesis Template.
      \item
        Export your LyX document to plain \LaTeX{} and move this file
        to the \LaTeX{} UTRGV Thesis Template Directory.
      \item
        Add \verb|\usepackage{docmute}| to preamble of ``main.tex'' file.
      \item
        Use the \verb|\input{}| command to include your document in the
        ``main.tex'' file.

  \end{enumerate}
\end{frame}

%------------------------------------------------
\section{Bibliography Management}
\begin{frame}[fragile]{Bibliography Management}
  The thesis template has a file ``references.bib'' you can use
  to manage your bibliographic references. This file is in the BibTeX format.
  \url{https://en.wikipedia.org/wiki/BibTeX}\\

  You can populate this file by searching on Google Scholar or MathSciNet and
  clicking on the ``cite'' link below a search result to get the proper
  BibTeX-formatted entry.\\

  Entries in file can then be referenced in your paper using
  the \verb|\cite| command.


\end{frame}


\section{Most Important Tip}
\begin{frame}
\frametitle{Most Important Tip}
{\LARGE\centering
    Focus on the content of your paper!!! \\
}
\end{frame}

\end{document}
